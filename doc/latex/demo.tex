\chapter{Usage}

The package comes with a demonstration program in the {\tt demo} directory
showing how to use the {\tt ITR} library. The  example can be built as follows
\begin{verbatim}
> cd /path/to/ITR-build/demo
> make demo
> ./demo
\end{verbatim}
Generally, one needs to {\tt -I/path/to/ITR-install/include/itr} and
{\tt -L/path/to/ITR-install/lib -litr} to the compile command. 

To use the library, one needs to include the header file {\tt ITR.h}. Inside
{\tt main()}, an {\tt ITR::ITR} instance is created by passing two arguments to
its constructor. The first argument is a {\tt std::string} object specifying the
path to the input csv file. The second argument is an unsigned integer specifying
the search depth. 

For the csv input file, the first line must be a header. The first column is the
subject identifier, followed by continuous variables (labeled as {\tt cont*}), ordinal
variables (labeled as {\tt ord*}), nominal variables (labeled as {\tt nom*}),
actions (labeled as {\tt A*}), responses (labeled as {\tt Y*}), and condition
probablity $P(A = 1 | X)$. The ITR library is case insensitive to these labels.

For the search depth, the valid values are 1, 2, and 3.   

The constructor of `ITR::ITR` will throw an exception if the input file does not
exist or the search depth is invalid. 

After the instance is created, one performs the search by invoking the {\tt run}
method. Here, one has the choice of running the search sequentially or in
parallel by passing an {\tt unsigned} integer {\tt nThreads} to the method. If
{\tt nThreads} is 1, the search is done sequentially. If {\tt nThreads} is greater
than 1, the search is done in parallel. 

Once the search completes, one can query the top {\tt n} results by calling the
{\tt report} method. An sample output of the {\tt demo} program looks like the following
\begin{verbatim}
 > ./demo --thread=8
 Loading input data ...
 Creating search engine with depth 3
 Searching 86131 choices ...
 Completed in 1.102359e-02 seconds using 8 threads
 Best 5 results: ...
 Value = 6.884978e+01
   X1 < 5.094425e+01
   X2 >= 5.253060e+01
   X7 not in {0, 2} 
 Value = 6.787278e+01
   X1 < 5.942735e+01
   X6 not in {2, 3} 
   X7 not in {2, 3} 
 Value = 6.744488e+01
   X1 < 5.094425e+01
   X7 not in {0, 2} 
   X8 not in {2} 
 Value = 6.742991e+01
   X1 < 4.372625e+01
   X5 < 3
   X6 not in {2, 3} 
 Value = 6.740870e+01
   X1 < 5.094425e+01
   X6 not in {0, 2} 
   X8 not in {2} 
\end{verbatim}


