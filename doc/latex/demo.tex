\chapter{Using ITR}

This chapter describes how to use ITR as a C++ library and a R module. There
are three steps of using ITR:

\begin{enumerate}
\item Register the data set to be analyzed with ITR.
\item Create an analyzer object.
\item Perform the analysis and query the results.
\end{enumerate}

\section{Register data set}
Any data set wants to be analyzed must be registered by calling the function
%
\begin{lstlisting}
size_t register_data(const std::string &input)
\end{lstlisting}
%
The input specifies the path to the data set in the csv format. The function returns
an identifier of the data set which can be passed to the analyzer object discussed
later.

There are a few requirements on the csv input file. The first line of the file must
be a header. The first column is the subject identifier, followed by
continuous variables (labeled as {\tt cont*}), ordinal variables (labeled
as {\tt ord*}), nominal variables (labeled as {\tt nom*}), actions (labeled
as {\tt A*}), responses (labeled as {\tt Y*}), and condition probability
$P(A = a_i |X_i)$. The ITR library is case insensitive to these labels. An example of
the data set looks the following

\begin{lstlisting}[frame=]
ID,Cont1,Cont2,Cont3,Ord1,Ord2,Ord3,Nom1,Nom2,Nom3,A,Y,P(A=1|X)
1,20.0871,15.6859,135.029,4,2,0,4,3,3,0,31.2112,0.5
2,284.225,5.51997,74.278,0,3,3,2,3,3,0,20.8458,0.5
3,62.2001,75.9237,11.6757,1,3,3,4,1,3,0,94.7321,0.5
\end{lstlisting}

\section{Create analyzer object}
There are two types of analyer objects in ITR: comprehensive search (CS) and
angle-based classifier (ABC). The constructor for CS takes two parameter:
\begin{lstlisting}
CompSearch(unsigned depth, unsigned nthreads)
\end{lstlisting}
The first parameter specifies the depth of the search and the valid values are
1, 2, and 3. The second parameter specifies the number of threads to use. The value
is internally truncated up to the number of available cores in the system. If
the value is set to 1, the computation is performed sequentially. 

The constructor for ABC takes four parameters:
\begin{lstlisting}
AngleBasedClassifier(double c, double lambda, const std::string kernel,
                     unsigned threads)
\end{lstlisting}
The first parameter is a positive real number for the large-margin loss function.
The second parameter is the weight of the penalty term in the objective function.
The third parameter is a string object describing the kernel function and its
parameters. The valid values are: (1) {\tt rbf xxx} for the radial basis function
kernel where the kernel parameter $\sigma$ is set to {\tt xxx}. (2)
{\tt poly xxx yyy} for the polynomial kernel where {\tt xxx} is the shift and
{\tt yyy} is the degree of the polynomial. If {\tt yyy} is set to 1, it is simply
the linear kernel. The last parameter is the number of threads to use. 

\section{Perform analysis and query results}
To analyze any registered data set, one first calls the {\tt preprocess} function 
\begin{lstlisting}
void preprocess(size_t i)
\end{lstlisting}   
passing in the identifier of the data set returned by the {\tt registeer\_data}
function. Afterwards, the analysis can be performed by calling the {\tt run} method
of the analyzer object. For the CS object, the {\tt run} method takes no parameter.
For the ABC object, the {\tt run} method takes three parameters to configure the
underlying L-BFGS solver.
\begin{lstlisting}
int AngleBasedClassifier::run(size_t maxItr, size_t m, double eps)
\end{lstlisting}
The first parameter specifies the maximum number of iterations allowed for the
L-BFGS solver. The second parameter specifies the number of correction terms
saved by the L-BFGS solver. The third parameter is a positive value controlling
the accuracy of the solution. In terms of return values, value 0 indicates that
the solution is found within the allowed number of iterations; value 1 indicates
that no solution is found after the maximum allowed iterations; value -1
indicates error happens during the iteration. 

Once the {\tt run} method completes, one can query the results of the analysis.
For CS, the available choices are
\begin{lstlisting}
rVector CompSearch::score(size_t ntop)
\end{lstlisting}
This function reports the best scores. 
\begin{lstlisting}
uMatrix CompSearch::var(size_t ntop)
\end{lstlisting}
This function reports the variables associated with the best scores. 
\begin{lstlisting}
sVector cut(size_t i)
\end{lstlisting}
This function reports the cut value associated with the ith best score. 
\begin{lstlisting}
sVector dir(size_t i) 
\end{lstlisting}
This function reports the cut direction associated with the ith best score.

For ABC, the available choice is
\begin{lstlisting}
rVector AngleBasedClassifer::beta()
\end{lstlisting}
This function returns the solution to the nonlinear optimization problem. 

\section{C++ demo}

The package comes with a demonstration program in the {\tt demo} directory.
To build it, using the configurations in Section~\ref{sec:c}, one performs
the following steps
\begin{lstlisting}[language=bash, frame=]
> cd /path/to/build/demo
> make demo
\end{lstlisting}
For other use, one needs to include the header file {\tt ITR.h} in the source
code that uses the library, and needs to supply
{\tt -I/path/to/install/include/itr} and {\tt -L/path/to/install/lib -litr}
to compile the code and link against the library. The output of the {\tt demo}
prorgam looks like the following
\begin{lstlisting}
Score = 68.8498, X1 < 49.8351 (percentile 50), X2 >= 49.6823 (percentile 20), X7 not in 0 2 
Score = 67.8728, X1 < 58.8109 (percentile 60), X6 not in 2 3 , X7 not in 2 3 
Score = 67.4449, X1 < 49.8351 (percentile 50), X7 not in 0 2 , X8 not in 2 
Score = 67.4299, X1 < 42.1108 (percentile 40), X5 < 3 (4 out of 5), X6 not in 2 3 
Score = 67.4087, X1 < 49.8351 (percentile 50), X6 not in 0 2 , X8 not in 2 
0.932404
-0.125072
-0.0889801
0.102649
-0.00148621
0.00923544
0.0340478
-0.145428
...
\end{lstlisting}

\section{R demo}

Here is an R session demonstrating the usage of the package.

\begin{lstlisting}[language=R, showstringspaces=false]
> library('ITR')
Loading required package: Rcpp
> data_id <- register_data('sample100.csv')
> cs <- new(CompSearch, 3, 8)
> cs$preprocess(data_id)
> cs$run()
> cs$score(5)
[1] 68.84978 67.87278 67.44488 67.42991 67.40870
> cs$var(5)
     [,1] [,2] [,3]
[1,]    1    2    7
[2,]    1    6    7
[3,]    1    7    8
[4,]    1    5    6
[5,]    1    6    8
> cs$cut(2)
[[1]]
[1] "58.8109 (percentile 60)"

[[2]]
[1] "2 3 "

[[3]]
[1] "2 3 "

> cs$dir(2)
[[1]]
[1] " < "

[[2]]
[1] " not in "

[[3]]
[1] " not in "
> abc <- new(AngleBasedClassifier, 10.0, 1.0, "rbf 1e3", 1)
> abc$preprocess(data_id)
> abc$run(100, 10, 1e-1)
> abc$beta()
  [1]  9.368988e-01 -1.282671e-01 -1.012102e-01  9.934106e-02 -1.130552e-02
  [6]  5.073681e-03  2.694525e-02 -1.426019e-01  9.910970e-02 -1.102812e-01
 [11]  8.493024e-03  8.918804e-03 -2.457499e-02 -6.818580e-02 -1.817148e-02
 [16]  6.353913e-02  8.686820e-02  1.021895e-01  8.709502e-02 -2.069887e-02
 [21]  6.153464e-02 -6.866418e-02 -7.470917e-02  3.794699e-02  2.184204e-02
 [26]  8.245382e-02 -8.168155e-02 -1.002616e-02  3.766475e-02  5.580140e-02
 [31]  5.418532e-02 -9.802538e-02  1.065100e-01  1.443494e-01 -1.361683e-01
 [36]  1.331555e-01 -1.286964e-01  1.298254e-01  2.617285e-02 -3.220321e-02
 [41] -1.019195e-01 -9.550773e-02 -5.108990e-02 -1.372925e-01  4.591341e-02
 [46] -1.437009e-01 -3.149146e-02  7.162312e-02 -1.464040e-01  6.264910e-02
 [51] -9.430271e-02 -1.139210e-01  8.449190e-02 -1.335786e-01  6.590237e-02
 [56]  8.748744e-02 -6.655964e-02 -2.470679e-02  7.374848e-02  6.311262e-02
 [61]  4.016364e-02  6.806053e-02 -8.459916e-02 -1.378274e-02  1.249061e-02
 [66]  9.976777e-03 -3.141552e-02  1.155593e-01 -2.821169e-02  1.761921e-02
 [71] -1.388808e-02  1.677525e-01  8.383804e-02 -9.458604e-02  1.243435e-02
 [76]  7.769387e-02 -1.573087e-01 -8.478658e-02  7.738239e-02 -2.592789e-02
 [81] -5.793699e-02 -7.709801e-02  2.415769e-06 -4.766364e-02 -9.346931e-02
 [86] -1.048874e-01  1.023146e-01 -1.096255e-01  1.240425e-01  2.887337e-02
 [91]  1.165838e-01 -7.381607e-02  3.223406e-02 -6.719682e-02  1.093244e-01
 [96]  1.235799e-01 -1.006323e-02  1.145411e-01 -9.606964e-03  5.631287e-02
[101]  5.073803e-02
> 
\end{lstlisting}
